\documentclass{report}


\def\letus{%
    \mathord{\setbox0=\hbox{$\exists$}%
             \hbox{\kern 0.125\wd0%
                   \vbox to \ht0{%
                      \hrule width 0.75\wd0%
                      \vfill%
                      \hrule width 0.75\wd0}%
                   \vrule height \ht0%
                   \kern 0.125\wd0}%
           }%
}


\usepackage{graphicx}
\usepackage{cancel}
\usepackage{amsmath}
\usepackage{amsfonts}
\usepackage{amssymb}
\usepackage{ragged2e}
\usepackage{pgfplots}


\usepackage{setspace}
\setstretch{1.7}
\everymath{\displaystyle}
\usepackage{nopageno}
\usepackage[utf8]{inputenc}
\usepackage[english,russian]{babel}
\usepackage[a5paper, left=0.4cm, top=0.5cm, right=0.4cm, bottom=0.5cm]{geometry}

\begin{document}

\text{\textbf{14.} Исследовать на равномерную сходимость}\
$ \sum\limits_{n = 1}^{\infty} \frac{e^{-n^{2} x^{2}}}{n^{2}} $
\text{на} $ \mathbb{R}.$

\text{\textbf{15.} Разложить в ряд Тейлора функцию}\
$ \pi + \int\limits_{\frac{\pi}{2}}^{x} \frac{\cos{t}^2}{t - \frac{\pi}{2}}\ dt $
\text{в окрестности точки}\ $ x_0 = \frac{\pi}{2} $\\
\text{и найти радиус сходимости полученного ряда.}

\text{\textbf{1.} Найти пределы}\
$ \lim\limits_{x \to 0} {\lim\limits_{y \to 0}{u}}, 
\lim\limits_{y \to 0} {\lim\limits_{x \to 0}{u}},
\lim\limits_{{\substack{x \to 0 \\ y \to 0}}}{u}, $
\text{if}\ $u = x + y\sin{\frac{1}{x}}. $\\
\indent
$ \lim\limits_{x \to 0} {\lim\limits_{y \to 0}{u}} = 0; $
$ \lim\limits_{y \to 0} {\lim\limits_{x \to 0}}{u}\ \nexists$
\text{т. к.}\ $ \lim\limits_{x \to 0} {\sin{\frac{1}{x}}} \nexists; $\\
\indent
$ \lim\limits_{{\substack{x \to 0 \\ y \to 0}}}{u} = 0 $
\text{т. к.}\ $\lim\limits_{\substack{x \to 0 \\ y \to 0}}{ x + y\sin{\frac{1}{x}}} \sim \lim\limits_{\substack{x \to 0 \\ y \to 0}} {y\sin{\frac{1}{x}}} = 0 $

\text{(произведение ограниченной функции на бесконечно малое).}\\

\text{\textbf{2.} Найти}\
$ xz \frac{\partial z}{\partial x} 
+ yz\frac{\partial z}{\partial y}, $
\text{if}\ 
$ z = \sqrt{xy + \varphi(\frac{y}{x})} .$

$ \letus u = z^2,\ $
$ \frac{\partial z}{\partial x} =
\frac{\partial \sqrt{u}}{\partial u} \frac{\partial u}{\partial x} $
$ = \frac{\frac{\displaystyle \partial}{\displaystyle \partial x}
(xy + \varphi(\frac{x}{y})) }{2z} $
$ = \frac{y + \displaystyle \frac{\varphi'(\frac{x}{y})}{y}}{2z} $

$ \frac{\partial z}{\partial y} = 
\frac{\partial \sqrt{u}}{\partial u} \frac{\partial u}{\partial y} $
$ = \frac{\frac{\displaystyle \partial}{\displaystyle \partial y}
(xy + \varphi(\frac{x}{y})) }{2z} $
$ = \frac{x - \displaystyle \frac{x \varphi'(\frac{x}{y})}{y^2}}{2z} $

$ \rightarrow \frac{xz(y + \displaystyle \frac{\varphi'(\frac{x}{y})}{y})}{2z} + $
$ \frac{yz(x - \displaystyle \frac{x \varphi'(\frac{x}{y})}{y^2})}{2z} $
$ = \frac{2xyz + \cancel{\displaystyle \frac{xz \varphi'(\frac{x}{y})}{y}}
- \cancel{\displaystyle \frac{xz \varphi'(\frac{x}{y})}{y}}
} {2z} $
$ = xy.$

\text{\textbf{3.} Найти экстремумы функции}\
$ z = \frac{1}{x} + \frac{1}{y}\ $ \text{при}\
$ \frac{1}{x^2} + \frac{1}{y^2} = \frac{1}{a^2}. $

$ L = \frac{1}{x} + \frac{1}{y} + \lambda(\frac{1}{x^2} + \frac{1}{y^2}
- \frac{1}{a^2});\ x \neq 0,\ y \neq 0,\ a \neq 0 $\\
\indent
$ \begin{cases}
    L'_x = -\frac{x + 2\lambda}{x^3} = 0 \rightarrow x = -2\lambda\\
    L'_y = -\frac{y + 2\lambda}{y^3} = 0 \rightarrow y = -2\lambda\\
    L'_{\lambda} = \frac{1}{x^2} + \frac{1}{y^2} - \frac{1}{a^2} = 0
\end{cases} $\\
\indent
$ \frac{1}{4\lambda^2} + \frac{1}{4\lambda^2} - \frac{1}{a^2} = 0,\
\lambda = \pm \frac{a \sqrt{2}}{2} \rightarrow y = x = \mp a\sqrt{2} $

$ L''_{xx}  = \frac{2x + 6\lambda}{x^4},\ L''_{yy} = \frac{2y + 6\lambda}{y^4},\
L''_{\lambda \lambda} = 0,\ L''_{x \lambda} = \frac{-2}{x^3},\ L''_{y \lambda}
= \frac{-2}{y^3} $\\
$ \mathcal{H} = \begin{vmatrix}
    0 & \frac{-2}{x^3} & \frac{-2}{y^3}\\
    \frac{-2}{x^3} & \frac{2x + 6\lambda}{x^4} & 0 \\
    \frac{-2}{y^3} & 0 & \frac{2y + 6\lambda}{y^4}\\
\end{vmatrix} $\\
\text{в точке}\ $ (-a\sqrt{2}, -a\sqrt{2}),\ \mathcal{H} = -\frac{1}{2\sqrt{2}a^9}; $
\text{в точке}\ $ (a\sqrt{2}, a\sqrt{2}),\ \mathcal{H} = \frac{1}{2\sqrt{2}a^9}. $

\text{if}\ $ a > 0 \rightarrow  -\frac{1}{2\sqrt{2}a^9} < 0 \rightarrow $
\text{точка}\ $ (-a\sqrt{2}, -a\sqrt{2}) $ \text{-- min},\
\text{а}\ $ (a\sqrt{2}, a\sqrt{2}) $ \text {-- max}.

\text{if}\ $ a < 0 \rightarrow  -\frac{1}{2\sqrt{2}a^9} > 0 \rightarrow $
\text{точка}\ $ (-a\sqrt{2}, -a\sqrt{2}) $ \text{-- max},\
\text{а}\ $ (a\sqrt{2}, a\sqrt{2}) $ \text {-- min}.

\text{\textbf{4.} Оператор Лапласа}\
$ \frac{\partial^2 u}{\partial x^2} + \frac{\partial^2 u}{\partial y^2} $
\text{преобразовать к полярным координатам.}


\text{\textbf{6.} В двойном интеграле}\
$ \iint\limits_{\mathbb{D}} f(x,y)\ dx dy $
\text{расставить пределы интегрирования}

\text{в том и другом порядке, if}\
$ \mathbb{D} = \{x^2 + y^2 \leq 1,\ x + y - 1 \leq 0,\ y \geq 0 \}. $\\
$ \begin{tikzpicture}
  \begin{axis}[axis equal, axis lines=middle, xlabel={$x$}, ylabel={$y$}, domain=-1.2:1.2, samples=100]
    \addplot[draw=black, thick, fill=none, draw opacity=1, domain=0:360] ({cos(x)}, {sin(x)});
    \addplot[domain=-0.6:1, color=black] {1 - x};
    \addplot[fill=gray!50, opacity=0.5, domain=-0:1] {1 - x} \closedcycle;
    \addplot[fill=gray!50, opacity=0.5, domain=-1:0, restrict y to domain=0:sqrt(1-x^2)] {ifthenelse(y >= 0, sqrt(1-x^2), nan)} \closedcycle;
  \end{axis}
\end{tikzpicture} $\\
$ \int\limits_{0}^{1} dy \int\limits_{\sqrt{1-y^2}}^{1-y} f(x,y) dx $
\text{по x разобьется на 2}\ $ \int\limits_{0}^{1} dx \int\limits_{0}^{1-x} f(x,y) dy +
\int\limits_{-1}^{0} dx \int\limits_{0}^{\sqrt{1-x^2}} f(x,y) dy $

\end{document}
