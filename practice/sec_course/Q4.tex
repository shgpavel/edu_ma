\documentclass{report}


\def\letus{%
    \mathord{\setbox0=\hbox{$\exists$}%
             \hbox{\kern 0.125\wd0%
                   \vbox to \ht0{%
                      \hrule width 0.75\wd0%
                      \vfill%
                      \hrule width 0.75\wd0}%
                   \vrule height \ht0%
                   \kern 0.125\wd0}%
           }%
}


\usepackage{graphicx}
\usepackage{cancel}
\usepackage{amsmath}
\usepackage{amsfonts}
\usepackage{amssymb}
\usepackage{ragged2e}
\usepackage{pgfplots}
\pgfplotsset{compat=1.18}
\usepackage{tikz}

\usepackage{setspace}
\setstretch{1.6}
\everymath{\displaystyle}
\usepackage{nopageno}
\usepackage[utf8]{inputenc}
\usepackage[english,russian]{babel}
\usepackage[a5paper, left=0.4cm, top=0.5cm, right=0.4cm, bottom=0.5cm]{geometry}

\begin{document}

\text{\textbf{14.} Исследовать на равномерную сходимость}\
$ \sum\limits_{n = 1}^{\infty} \frac{e^{-n^2 x^2}}{n^2} $
\text{на} $ \mathbb{R}.$

$ \left| \frac{e^{-n^{2} x^{2}}}{n^2} \right| \leq \frac{1}{n^2} $
\text{т. к.} 
$ \sum\limits_{n = 1}^{\infty} \frac{1}{n^2} $
\text{сходится на}\ $ \mathbb{R}, $ 
\text{то исходный ряд}

\text{по признаку Вейерштрасса равномерно сходится на}\
$ \mathbb{R}. $~/.local/state/nvim/swap//%home%main%de
v%git%edu_matan%practice%sec_course%thisem.tex.swp

\text{\textbf{15.} Разложить в ряд Тейлора функцию}\
$ \pi + \int\limits_{\frac{\pi}{2}}^{x} \frac{\cos^2{t}}{t - \frac{\pi}{2}}\ dt $
\text{в окрестности точки}\ $ x_0 = \frac{\pi}{2} $
\indent
\text{и найти радиус сходимости полученного ряда.}

$ f' = \frac{\cos^2{x}}{x - \displaystyle \frac{\pi}{2}} $

$ \cos^2{x} =  1 + \sum\limits_{n = 1}^{\infty} \frac{{(-1)}^n (2)^{2n-1} {\left(x - 
\displaystyle \frac{\pi}{2}\right)}^{2n}}{(2n)!} $
$ \cos{2x} = \sum\limits_{n = 0}^{\infty} \frac{{(-1)}^n {(2x - \frac{\pi}{2})}^{2n} }{(2n)!};\ $

$ {(x - \frac{\pi}{2})}^{-1} =
\sum\limits_{n = 1}^{\infty} -{\left(\frac{2}{\pi}\right)}^n {\left(x - \frac{\pi}{2}\right)}^{n - 1} $

$ f = \int\limits_{1}^{x} \left(
    \sum\limits_{n = 1}^{\infty} -{\left(\frac{\pi}{2}\right)}^n 
    {\left(x - \frac{\pi}{2}\right)}^{1-n}
    + \sum\limits_{n = 1}^{\infty} \frac{\pi^n {\left(x - 
    \displaystyle \frac{\pi}{2}\right)}^{n+1}}{2 (2n)!}
\right)dx $

$ = \sum\limits_{n = 1}^{\infty} \int\limits_{1}^{x} -{\left(\frac{\pi}{2}\right)}^n 
    {\left(x - \frac{\pi}{2}\right)}^{1-n} dx
    + \sum\limits_{n = 1}^{\infty} \int\limits_{1}^{x} \frac{\pi^n {\left(x - 
    \displaystyle \frac{\pi}{2}\right)}^{n+1}}{2 (2n)!} dx
$

$ =  \sum\limits_{n = 0}^{\infty} -{(\frac{\pi}{2})^{n+1}\
\frac{ {(x - \frac{\pi}{2})}^{3n + 1} }{(3n + 1) ((2n)!)^2}}. $

$ R = \lim\limits_{n \to \infty} \frac{{(\frac{\pi}{2})}^{n+1}\
    \frac{1}{(3n + 1) ((2n)!)^2}}{{(\frac{\pi}{2})}^{n+2}\
    \frac{1}{(3n + 4) ((2n + 2)!)^2}} = \infty $

\text{\textbf{1.} Найти пределы}\
$ \lim\limits_{x \to 0} {\lim\limits_{y \to 0}{u}}, 
\lim\limits_{y \to 0} {\lim\limits_{x \to 0}{u}},
\lim\limits_{{\substack{x \to 0 \\ y \to 0}}}{u}, $
\text{if}\ $u = x + y\sin{\frac{1}{x}}. $\\
\indent
$ \lim\limits_{x \to 0} {\lim\limits_{y \to 0}{u}} = 0; $
$ \lim\limits_{y \to 0} {\lim\limits_{x \to 0}}{u}\ \nexists$
\text{т. к.}\ $ \lim\limits_{x \to 0} {\sin{\frac{1}{x}}} \nexists; $\\
\indent
$ \lim\limits_{{\substack{x \to 0 \\ y \to 0}}}{u} = 0 $
\text{т. к.}\ $\lim\limits_{\substack{x \to 0 \\ y \to 0}}{ x + y\sin{\frac{1}{x}}} \sim \lim\limits_{\substack{x \to 0 \\ y \to 0}} {y\sin{\frac{1}{x}}} = 0 $

\text{(ограниченная функция на б. м.)}

\text{\textbf{2.} Найти}\
$ xz \frac{\partial z}{\partial x} 
+ yz\frac{\partial z}{\partial y}, $
\text{if}\ 
$ z = \sqrt{xy + \varphi(\frac{y}{x})} .$

$ \letus u = z^2,\ $
$ \frac{\partial z}{\partial x} =
\frac{\partial \sqrt{u}}{\partial u} \frac{\partial u}{\partial x} $
$ = \frac{\frac{\displaystyle \partial}{\displaystyle \partial x}
(xy + \varphi(\frac{x}{y})) }{2z} $
$ = \frac{y + \displaystyle \frac{\varphi'(\frac{x}{y})}{y}}{2z} $

$ \frac{\partial z}{\partial y} = 
\frac{\partial \sqrt{u}}{\partial u} \frac{\partial u}{\partial y} $
$ = \frac{\frac{\displaystyle \partial}{\displaystyle \partial y}
(xy + \varphi(\frac{x}{y})) }{2z} $
$ = \frac{x - \displaystyle \frac{x \varphi'(\frac{x}{y})}{y^2}}{2z} $

$ \rightarrow \frac{xz(y + \displaystyle \frac{\varphi'(\frac{x}{y})}{y})}{2z} + $
$ \frac{yz(x - \displaystyle \frac{x \varphi'(\frac{x}{y})}{y^2})}{2z} $
$ = \frac{2xyz + \cancel{\displaystyle \frac{xz \varphi'(\frac{x}{y})}{y}}
- \cancel{\displaystyle \frac{xz \varphi'(\frac{x}{y})}{y}}
} {2z} $
$ = xy.$

\text{\textbf{3.} Найти экстремумы функции}\
$ z = \frac{1}{x} + \frac{1}{y}\ $ \text{при}\
$ \frac{1}{x^2} + \frac{1}{y^2} = \frac{1}{a^2}. $

$ L = \frac{1}{x} + \frac{1}{y} + \lambda(\frac{1}{x^2} + \frac{1}{y^2}
- \frac{1}{a^2});\ x \neq 0,\ y \neq 0,\ a \neq 0 $\\
\indent
$ \begin{cases}
    L'_x = -\frac{x + 2\lambda}{x^3} = 0 \rightarrow x = -2\lambda\\
    L'_y = -\frac{y + 2\lambda}{y^3} = 0 \rightarrow y = -2\lambda\\
    L'_{\lambda} = \frac{1}{x^2} + \frac{1}{y^2} - \frac{1}{a^2} = 0
\end{cases} $\\
\indent
$ \frac{1}{4\lambda^2} + \frac{1}{4\lambda^2} - \frac{1}{a^2} = 0,\
\lambda = \pm \frac{a \sqrt{2}}{2} \rightarrow y = x = \mp a\sqrt{2} $

$ L''_{xx}  = \frac{2x + 6\lambda}{x^4},\ L''_{yy} = \frac{2y + 6\lambda}{y^4},\
L''_{\lambda \lambda} = 0,\ L''_{x \lambda} = \frac{-2}{x^3},\ L''_{y \lambda}
= \frac{-2}{y^3} $\\
\indent
$ \mathcal{H} = \begin{vmatrix}
    0 & \frac{-2}{x^3} & \frac{-2}{y^3}\\
    \frac{-2}{x^3} & \frac{2x + 6\lambda}{x^4} & 0 \\
    \frac{-2}{y^3} & 0 & \frac{2y + 6\lambda}{y^4}\\
\end{vmatrix} $\\

\text{в точке}\ $ (-a\sqrt{2}, -a\sqrt{2}),\ \mathcal{H} = -\frac{1}{2\sqrt{2}a^9}; $
\text{в точке}\ $ (a\sqrt{2}, a\sqrt{2}),\ \mathcal{H} = \frac{1}{2\sqrt{2}a^9}. $

\text{if}\ $ a > 0 \rightarrow  -\frac{1}{2\sqrt{2}a^9} < 0 \rightarrow $
\text{точка}\ $ (-a\sqrt{2}, -a\sqrt{2}) $ \text{-- min},\
\text{а}\ $ (a\sqrt{2}, a\sqrt{2}) $ \text {-- max}.

\text{if}\ $ a < 0 \rightarrow  -\frac{1}{2\sqrt{2}a^9} > 0 \rightarrow $
\text{точка}\ $ (-a\sqrt{2}, -a\sqrt{2}) $ \text{-- max},\
\text{а}\ $ (a\sqrt{2}, a\sqrt{2}) $ \text {-- min}.

\text{\textbf{4.} Оператор Лапласа}\
$ \frac{\partial^2 u}{\partial x^2} + \frac{\partial^2 u}{\partial y^2} $
\text{преобразовать к полярным координатам.}

$ \begin{cases}
    \rho = \sqrt{x^2 + y^2}\\
    \varphi = \arctg(\frac{y}{x})\\
    w = u
\end{cases} $

$ \begin{cases}
    d\rho = \frac{x}{\sqrt{x^2 + y^2}} dx + \frac{y}{\sqrt{x^2 + y^2}} dy\\
    d\varphi = \frac{x}{x^2 + y^2} dy - \frac{y}{x^2 + y^2} dx\\
    dw = du
\end{cases} $
\indent
$ dw = w'_{\rho} d\rho + w'_{\varphi} d\varphi = u'_x dx + u'_y dy $

$ \begin{cases}
    u'_x = \frac{x}{\sqrt{x^2 + y^2}} w'_{\rho} - \frac{y}{x^2 + y^2} w'_{\varphi}\\
    u'_y = \frac{y}{\sqrt{x^2 + y^2}} w'_{\rho} + \frac{x}{x^2 + y^2} w'_{\varphi}
\end{cases} $
$ \implies $
$ \begin{cases}
    u'_x = \cos{\varphi} w'_{\rho} - \frac{\sin{\varphi}}{\rho} w'_{\varphi}\\
    u'_y = \sin{\varphi} w'_{\rho} + \frac{\cos{\varphi}}{\rho} w'_{\varphi}
\end{cases} $

$ u''_{x^{2}} = \left(\cos{\varphi} w'_{\rho} - \frac{\sin{\varphi}}{\rho} w'_{\varphi}\right)'_\rho
\rho'_x + \left( \cos{\varphi} w'_{\rho} - \frac{\sin{\varphi}}{\rho} w'_{\varphi}\right)'_\varphi \varphi'_x = $

$ = w''_{\rho^2}\cos^2{\varphi} - 2 w''_{\rho\varphi} \frac{\sin{\varphi} \cos{\varphi}}{\rho} + $
$ w''_{\varphi^2} \frac{\sin^2{\varphi}}{\rho^2} $
$ + w'_\rho \frac{\sin^2{\varphi}}{\rho} + w'_\varphi \frac{\sin{\varphi}\cos{\varphi}}{\rho^2} $

$ u''_{y^{2}} = \left(\sin{\varphi} w'_{\rho} + \frac{\cos{\varphi}}{\rho} w'_{\varphi}\right)'_\rho \rho'_y $
$ + \left(\sin{\varphi} w'_{\rho} + \frac{\cos{\varphi}}{\rho} w'_{\varphi}\right)'_\varphi \varphi'_y = $

$ = w''_{\rho^2}\sin^2{\varphi} + 2 w''_{\rho\varphi} \frac{\sin{\varphi} \cos{\varphi}}{\rho} + w''_{\varphi^2}
\frac{\cos^2{\varphi}}{\rho^2} + w'_\rho \frac{\cos^2{\varphi}}{\rho} - w'_\varphi \frac{\sin{\varphi}\cos{\varphi}}{\rho^2} $

$ u''_{x^2} + u''_{y^2} = w''_{\rho^2} + \frac{1}{\rho} w'_\rho + \frac{1}{\rho^2} w''_{\varphi^2}. $

\text{\textbf{5.} Исследовать на равномерную сходимость интеграл}\
$ \int\limits_{1}^{+\infty} \frac{\ln^{3}{x}}{x^2 + \alpha^4} dx $
\text{на множестве}\ $ \mathbb{R}. $

$ \left| \frac{\ln^3{x}}{x^2 + \alpha^4} \right| \leq \frac{1}{x^{\frac{3}{2}}}, $
\text{т. к.}\ $ \lim\limits_{x \to \infty} {\frac{(\ln^3{x}) x^{\frac{3}{2}}}
{x^2 + \alpha^4}} = \lim\limits_{x \to \infty} {\frac{\ln^3{x}}{\sqrt{x}}} $
$ = \lim\limits_{x \to \infty} {\frac{6\ln^2{x}}{\sqrt{x}}} = \ldots  = 0. $

\text{По признаку Вейерштрасса т. к.}\
$ \int\limits_{1}^{+\infty} \frac{1}{x^{\frac{3}{2}}} dx $
\text{-- сходится}
$ \to $ \text{исходный интеграл}\\
\indent
\text{равномерно сходится на}\ $ \mathbb{R}. $

\text{\textbf{6.} В двойном интеграле}\
$ \iint\limits_{\mathbb{D}} f(x,y)\ dx dy $
\text{расставить пределы интегрирования}

\text{в том и другом порядке, if}\
$ \mathbb{D} = \{x^2 + y^2 \leq 1,\ x + y - 1 \leq 0,\ y \geq 0 \}. $\\

$ \begin{tikzpicture}
  \begin{axis}[axis equal, axis lines=middle, xlabel={$x$}, ylabel={$y$}, domain=-1.2:1.2, samples=100]
    \addplot[draw=black, thick, fill=none, draw opacity=1, domain=0:360] ({cos(x)}, {sin(x)});
    \addplot[domain=-0.6:1, color=black] {1 - x};
    \addplot[fill=gray!50, opacity=0.5, domain=0:1] {1 - x} \closedcycle;
    \addplot[fill=gray!50, opacity=0.5, domain=90:180] ({cos(x)}, {sin(x)}) \closedcycle;
  \end{axis}
\end{tikzpicture} $\\
\indent
$ \int\limits_{0}^{1} dy \int\limits_{-\sqrt{1-y^2}}^{1-y} f(x,y) dx, $
 $ \int\limits_{0}^{1} dx \int\limits_{0}^{1-x} f(x,y) dy +
\int\limits_{-1}^{0} dx \int\limits_{0}^{\sqrt{1-x^2}} f(x,y) dy .$

\text{\textbf{7.} Вычислить полную поверхность тела, ограниченного сферой}\
$ x^2 + y^2 + z^2 = 3a^2 $

\text{и параболоидом} $ x^2 + y^2 = 2az,\ z \geq 0. $

\newpage

\text{\textbf{8.} Найти момент инерции прямого круглого однородного цилиндра}

\text{(радиус основания $ R $, высота $ H $) относительно диаметра его среднего сечения.}

$
\begin{tikzpicture}
\begin{axis}[
    hide axis,
    view={130}{30},
    axis equal,
    ticks=none
]

\addplot3 [surf, opacity=0.6, shader=flat, samples=50, domain=0:360, colormap/blackwhite, y domain=0:2] ({2*cos(x)}, {2*sin(x)}, y);

  

\draw (axis cs: 0,0,1) circle [radius=2];

\draw[dashed] (axis cs: 0,0,1) -- (axis cs: 4,0,1);
\draw[dashed] (axis cs: 0,0,1) -- (axis cs: -4,0,1);

\filldraw (axis cs: 0,0,1) circle (2pt) node[below] {$O$};
\filldraw (axis cs: -3.5,0,1) circle (2pt) node[below] {$O'$};

\draw (axis cs: 0,0,0) circle [radius=2];
\draw (axis cs: 0,0,2) circle [radius=2];

\end{axis}
\end{tikzpicture}  $

$ d^2 = {\left( z - \frac{R}{2}\right)}^2 + h^2;\ $
$ h = r\sin{\varphi};\ $
$ d^2 = {\left( z - \frac{H}{2}\right)}^2 + r^2 \sin^2{\varphi} $

$ I = \rho\int\limits_{0}^{H}dz \int\limits_{0}^{2\pi} d\varphi \int\limits_{0}^{R} d^2r dr $
$ = \rho\int\limits_{0}^{H}dz \int\limits_{0}^{2\pi} d\varphi \int\limits_{0}^{R} \left({\left( z - \frac{H}{2}\right)}^2 r + r^3\sin^2\varphi \right) dr = $

$ = \rho\int\limits_{0}^{H}dz \int\limits_{0}^{2\pi} \left( \frac{{\left( z - \frac{H}{2}\right)}^2 R^2}{2} + 
\frac{R^4 \sin^2{\varphi}}{4} \right) d\varphi = $

$ = \rho\int\limits_{0}^{H}dz \left[ \frac{{\left( z - \frac{H}{2}\right)}^2 R^2}{2} \varphi + 
\frac{R^4}{8} \left(\varphi - \frac{\sin{2\varphi}}{2}\right) \right]_{0}^{2\pi} $
$ = \rho\int\limits_{0}^{H} \left( {\left( z - \frac{H}{2}\right)}^2 R^2 \pi + 
\frac{R^4}{4} \pi \right) dz = $

$ = \rho \left[\frac{{\left( z - \frac{H}{2}\right)}^3 - R^2 \pi}{3} + \frac{R^4 \pi}{4} \right]_{0}^{H} $
$ = \rho \pi R^2 H \left( \frac{H^2}{24} + \frac{R^2}{4} \right). $

\end{document}
